 %%%%%%%%%%%%%%%%%%%%%%%%%%%%%%%%%%%%%%%%%
% "ModernCV" CV and Cover Letter
% LaTeX Template
% Version 1.11 (19/6/14)
%
% This template has been downloaded from:
% http://www.LaTeXTemplates.com
%
% Original author:
% Xavier Danaux (xdanaux@gmail.com)
%
% License:
% CC BY-NC-SA 3.0 (http://creativecommons.org/licenses/by-nc-sa/3.0/)
%
% Important note:
% This template requires the moderncv.cls and .sty files to be in the same 
% directory as this .tex file. These files provide the resume style and themes 
% used for structuring the document.
%
%%%%%%%%%%%%%%%%%%%%%%%%%%%%%%%%%%%%%%%%%


%----------------------------------------------------------------------------------------
%	PACKAGES AND OTHER DOCUMENT CONFIGURATIONS
%----------------------------------------------------------------------------------------

\documentclass[11pt,letterpaper,sans]{moderncv} % Font sizes: 10, 11, or 12; paper sizes: a4paper, letterpaper, a5paper, legalpaper, executivepaper or landscape; font families: sans or roman

\moderncvstyle{classic} % CV theme - options include: 'casual' (default), 'classic', 'oldstyle' and 'banking'
\moderncvcolor{blue} % CV color - options include: 'blue' (default), 'orange', 'green', 'red', 'purple', 'grey' and 'black'

\usepackage{geometry}
 \geometry{
 letterpaper,
 total={210mm,297mm},
 left=10mm,
 right=10mm,
 top=20mm,
 bottom=20mm,
 }
\usepackage{tikz}
\usepackage{datetime}

\newdateformat{monthyeardate}{%
  \monthname[\THEMONTH], \THEYEAR}

%----------------------------------------------------------------------------------------
%	NAME AND CONTACT INFORMATION SECTION
%----------------------------------------------------------------------------------------

\firstname{Tanvir Ahmed} % Your first name
\familyname{Khan} % Your last name


\mobile{+1 7347735372} % Phone number
\email{takh@umich.edu} % Email ID
\address{1630 Cram Circle Apt 10}{Ann Arbor, Michigan, 48105}

%----------------------------------------------------------------------------------------

\begin{document}
\begin{tikzpicture}[remember picture,overlay]
  \node[anchor=north, yshift=-0.25cm] at (current page.north) {\textit{Last updated: \monthyeardate\today}};
\end{tikzpicture}

\makecvtitle % Print the CV title

\section{Research Interests}

%\cvitem{}{My research interest lies in the intersection of compilers, operating systems, and computer architecture.}
\cvitem{Areas}{Compilers, Operating Systems, Computer Architecture}
\cvitem{Topics}{Profile-guided Optimization, Locality Optimization, Prefetching}

\section{Education}

\cventry{2017-present}{Ph.D.}{\textit{University of Michigan}}{Ann Arbor, Michigan, USA}{}{}
\cvlistdoubleitem{Computer Science and Engineering}{CGPA: 3.775/4.00}
\cvlistitem{Thesis: Profile-guided Locality Optimization for Data Center Applications}{}
\cvlistitem{Advisor: Prof. Baris Kasikci}{}%\newline

\cventry{2014-2017}{M.Sc.}{\textit{Bangladesh University of Engineering and Technology}}{Dhaka, Bangladesh}{}{}
\cvlistdoubleitem{Computer Science and Engineering}{CGPA: 3.75/4.00}
\cvlistitem{Thesis: Overcoming Throughput Degradation in Multi-Radio Cognitive Radio Networks}{}
\cvlistitem{Advisor: Dr. A. B. M. Alim Al Islam}{}%\newline

\cventry{2009-2014}{B.Sc.}{\textit{Bangladesh University of Engineering and Technology}}{Dhaka, Bangladesh}{}{}
\cvlistdoubleitem{Computer Science and Engineering}{CGPA: 3.97/4.00}
\cvlistitem{Thesis: Too Many Cooks Spoil the Broth: Augmenting Spectral Harvesting with Multiple Radios Can Make the Performance Worse!}{}
\cvlistdoubleitem{Advisor: Dr. A. B. M. Alim Al Islam}{Class Rank: 1/153}%\newline

\section{Selected Publications}

\cvitem{OSDI'21}{\textbf{Tanvir Ahmed Khan}, Ian Neal, Gilles Pokam, Barzan Mozafari, and Baris Kasikci, DMon: Efficient Detection and Correction of Data Locality Problems using Selective Profiling. in proceedings of the 15\textsuperscript{th} USENIX Symposium on Operating Systems Design and Implementation (OSDI), USENIX Association, 2021.}

\cvitem{ISCA'21}{\textbf{Tanvir Ahmed Khan}, Dexin Zhang, Akshitha Sriraman, Joseph Devietti, Gilles Pokam, Heiner Litz, and Baris Kasikci, Ripple: Profile-Guided Instruction Cache Replacement for Data Center Applications. in proceedings of the 48\textsuperscript{th} International Symposium on Computer Architecture (ISCA), 2021.}

\cvitem{FAST'21}{Ian Neal, Gefei Zuo, Eric Shiple, \textbf{Tanvir Ahmed Khan}, Youngjin Kwon, Simon Peter, and Baris Kasikci, Rethinking File Mapping for Persistent Memory. in proceedings of the 19\textsuperscript{th} USENIX Conference on File and Storage Technologies (FAST), USENIX Association, 2021.}

\cvitem{MICRO'20}{\textbf{Tanvir Ahmed Khan}, Akshitha Sriraman, Joseph Devietti, Gilles Pokam, Heiner Litz, and Baris Kasikci, I-SPY: Context-Driven Conditional Instruction Prefetching with Coalescing. in proceedings of the 53\textsuperscript{rd} IEEE/ACM International Symposium on Microarchitecture (MICRO), IEEE, 2020.}

\cvitem{IISWC'20}{Yuhan Chen, Jingyuan Zhu, \textbf{Tanvir Ahmed Khan}, and Baris Kasikci, CPU Microarchitectural Performance Characterization of Cloud Video Transcoding. in proceedings of the IEEE International Symposium on Workload Characterization (IISWC), IEEE, 2020.}

\cvitem{PLDI'19}{\textbf{Tanvir Ahmed Khan}, Yifan Zhao, Gilles Pokam, Barzan Mozafari, and Baris Kasikci, Huron: Hybrid False Detection and Repair. in proceedings of the 40\textsuperscript{th} ACM SIGPLAN Conference on Programming Language Design and Implementation (PLDI), ACM, Phoenix, Arizona, United States, pages 473-488, 2019.}

\cvitem{MobiSys'16}{\textbf{Tanvir Ahmed Khan} and A. B. M. Alim Al Islam, Poster: Overcoming Throughput Degradation in Multi-Radio Cognitive Radio Networks. in proceedings of the 14\textsuperscript{th} Annual International Conference on Mobile Systems, Applications, and Services (MobiSys) Companion, 2016, ACM, Singapore, Singapore, pages 41-41, 2016.}

\cvitem{WiMob'15}{\textbf{Tanvir Ahmed Khan}, Chowdhury Sayeed Hyder, and A. B. M. Alim Al Islam, Towards exploiting a synergy between cognitive and multi-radio networking. in proceedings of the 11\textsuperscript{th} IEEE International Conference on Wireless and Mobile Computing, Networking and Communications (WiMob), 2015, IEEE, Abu Dhabi, UAE, pages 370-377, 2015.}

\section{Awards and Honors}
\cvitem{2020}{Facebook Fellowship Finalist}
\cvitem{2019}{PLMW@PLDI 2019 Student Scholarship covering conference registration, accommodation, and travel.}
\cvitem{2017-2018}{Rollin M. Gerstacker Foundation Fellowships, University of Michigan.}
\cvitem{2018}{OSDI 2018 Student Grant covering conference registration, accommodation, and travel.}
\cvitem{2016}{MobiSys 2016 Student Scholarship covering conference registration, accommodation, and travel.}
\cvitem{2014}{Crest of Honor, Highest CGPA in the department, presented by BUET alumni association.}
\cvitem{2009-2014}{University Merit Scholarship, Bangladesh University of Engr and Tech.}
\cvitem{2009-2014}{Dean's List Scholarship, Bangladesh University of Engr and Tech.}%\newline

\section{Employment}

\cventry{2017-present}{Research Assistant}{\textit{University of Michigan}}{Ann Arbor, Michigan, USA}{}{}
\cvlistitem{Advisor: Baris Kasikci}{}
\cvlistitem{Electrical Engineering and Computer Science Department.}{}%\newline

\cventry{Summer 2020}{Software Engineer Intern}{\textit{Facebook}}{Menlo Park, California, USA}{}{}
\cvlistitem{Mentor: Maksim Panchenko}
\cvlistitem{Binary Optimization and Layout Tool (BOLT) Team}{}%\newline

\cventry{Summer 2019}{Research Intern}{\textit{Microsoft}}{Redmond, Washington, USA}{}{}
\cvlistitem{Mentor: Gagan Gupta and Rathijit Sen}
\cvlistitem{Azure Hardware Research Group}{}%\newline

\cventry{Winter 2019}{Graduate Student Instructor}{\textit{University of Michigan}}{Ann Arbor, Michigan, USA}{}{}
\cvlistitem{Electrical Engineering and Computer Science Department.}{}
\cvlistitem{Primary Instructor: Thomas Wenisch}{}
\cvlistitem{Course: Parallel Computer Architecture.}{}% \newline

\cventry{2014-2017}{Lecturer}{\textit{Bangladesh University of Engineering and Technology}}{Dhaka, Bangladesh}{}{}
\cvlistitem{Department of Computer Science and Engineering.}{}
\cvlistitem{Courses Taught: Operating Systems, Computer Architecture, Compilers.}{}% \newline

\iffalse
  \cventry{2015-2016}{Part-time Lecturer}{\textit{Military Institute of Science and Technology}}{Dhaka, Bangladesh}{}{}
  \cvlistitem{Department of Computer Science and Engineering.}{}
  \cvlistitem{Courses Taught: Compilers Lab, Digital Electronics and Pulse Techniques Lab. \newline}{}
\fi

\section{Talks}
\cventry{2020}{I-SPY: Context-Driven Conditional Instruction Prefetching with Coalescing}{}{}{}{}
\cvlistitem{IEEE/ACM International Symposium on Microarchitecture (MICRO), October 2020}{}
%\cvlistitem{Students' Forum Speaker, International Conference on Networking, Systems and Security, December 2020}{}

\cventry{2019}{Huron: Hybrid False Sharing Detection and Repair}{}{}{}{}
\cvlistitem{Microsoft C++ Team, August 2019}{}
\cvlistitem{ACM SIGPLAN Conference on Programming Language Design and Implementation (PLDI), June 2019}{}
\cvlistitem{Azure Hardware Research Group, May 2019}{}

\cventry{2018}{Overcoming Throughput Degradation in Multi-Radio Cognitive Radio Networks}{}{}{}{}
\cvlistitem{Intel Labs Wireless Networking Research Group, May 2018}{}

%\newpage
\section{Selected Research Projects}

\cventry{2020-Present}{Profile-guided Binary Layout Optimization for Linux Kernel}{}{}{}{Many data center applications suffer from frequent instruction cache misses costing millions of dollars. While existing profile-guided optimization mechanisms can improve the instruction locality of user-mode instructions for these applications, no such technique exists for kernel-mode instructions. In this work, we first characterize the high percentage of instruction cache misses coming from kernel model instructions. Then, we enhance the state-of-the-art profile-guided optimization tool, BOLT to rewrite the Linux Kernel binary.}
\cventry{2019-Present}{Context-Driven Conditional Instruction Prefetching with Coalescing for Data Center Applications}{}{}{}{Modern data center applications exhibit large instruction footprints, leading to frequent instruction cache misses increasing cost and degrading data center performance and energy efficiency. To overcome this performance degradation via prefetching, we first investigate the challenges of effective instruction prefetching. We then use insights derived from our investigation to develop I-SPY, a novel profile-guided prefetching technique. Specifically, I-SPY proposes context-driven conditional prefetching and prefetch coalescing. For nine data center applications, I-SPY provides an average of 15.5\% speedup and 95.9\% reduction in instruction cache misses outperforming the state-of-the-art prefetching technique by 22.5\%.}
\cventry{2019-Present}{Performance Characteristics of Managed Workloads}{}{}{}{Many modern applications are written in managed code to achieve type safety, portability, security, and automatic memory management. While these features provide necessary abstractions for quick and safe software development, computer architectural aspects of these features are not well studied in the literature. In this study, we investigate the characteristics of these managed workloads. We identify the key bottlenecks that significantly hurt the performance of these workloads on modern hardware. We also correlate these bottlenecks on different aspects of managed runtime including common shared libraries, garbage collection, and just in time code generation.}
\cventry{2018-Present}{Continuous Locality Optimization}{}{}{}{Locality of reference is a key factor affecting a program's performance. However, very few programs exhibit good data locality during execution. In this project, we propose Erie, the first methodology that can continuously monitor executions in production for locality optimizations, with negligible overhead and without requiring special hardware. Using a hybrid approach, Erie efficiently identifies locality problems in production, and addresses them in house. Erie can reason about how its optimizations affect unobserved executions, as well as their interaction with underlying hardware mechanisms. Our evaluation shows that Erie's overhead is less than 1.36\% on average, and its optimizations provide up to 53.14\% (16.83\% on average) speedup, compared to a baseline that uses the highest level of compiler optimization. Erie provides up to 17.48\% (6.64\% on average) speedup on PostgreSQL, one of the most popular open-source data-base management systems.}
\cventry{2017-2019}{Automatic False Sharing Elimination}{}{}{}{Parallel programs on shared memory multiprocessors suffer from a slowdown of an order of magnitude due to false sharing. False sharing occurs when multiple threads demand exclusive access to a cache line even though their data locate on non-overlapping portions of the cache line. In this project, we proposed Huron, a hybrid in-house/in-production false sharing detection and repair system. Huron detects and repairs as much false sharing as it can in-house, and relies on its lightweight in-production mechanism for remaining cases. The key idea behind Huron's in-house false sharing repair is to group together data that is accessed by the same set of threads, to shift falsely-shared data to different cache lines. Huron's in-house repair technique can generalize to previously-unobserved inputs. Our evaluation shows that Huron can detect more false sharing bugs than all state-of-the-art techniques, and with a lower overhead. Huron improves runtime performance by 3.82$\times$ on average (up to 11$\times$), which is 2.11-2.27$\times$ better than the state of the art.}
%\cventry{2017-2018}{Feedback Directed Parallelization for Heterogeneous Systems}{}{}{}{Assigning computations to the correct processing unit is difficult because performance on heterogeneous systems maybe dominated by subtle effects that are hard to measure or anticipate through static analysis. In this project, we leveraged performance monitoring unit counters  to predict a computation's performance on different machines.}
\cventry{2013-2017}{Performance Analysis for Multi-Radio Cognitive Radio Networks}{}{}{}{In this project, we analyzed several performance metrics (throughput, delay, packet-loss, etc.) in multi-radio cognitive radio networks. Consequently, we proposed a feedback-based multi-radio exploitation approach to overcome throughput degradation in multi-radio cognitive radio networks.}

%\section{Selected Course Projects}

%\cvitem{-}{Implementation of Machine Learning Algorithms (Decision tree, K-nearest neighbor, Bayesian inference, Neural networks) from scratch.}
%\cvitem{-}{Compiler for a Subset of Pascal Language.}
%\cvitem{-}{Saint Basil's Cathedral using OpenGL.}
%\cvitem{-}{Implementation of OSI Layers over simulated Physical layer written in Java.}
%\cvitem{-}{Complete hardware implementation of a two-stage pipelined, micro-programmed microprocessor supporting 28 basic instructions.}

%\section{Industry Research Collaboration}

%\cventry{2016-2017}{Fraud Detection on Mobile Advertisement}{}{}{}{In collaboration with Widespace.}
%\cventry{2015-2017}{Early Detection of Terrorism from Social Media}{}{}{}{In collaboration with Infinity Technology and Bangladesh Police.}
%\cventry{2015-2016}{Target Audience Finding in Internet Advertisement}{}{}{}{In collaboration with Mindshare Bangladesh.}

\section{Selected Services}
\cvitem{2021}{External Review Committee Member for ASPLOS'21, Shadow Program Committee Member for EuroSys'21}
\cvitem{2020}{Artifact Evaluation Committee Member for PLDI'20, MLSys'20, and ASPLOS'20}
\cvitem{2019}{Artifact Evaluation Committee Member for SOSP'19, Student Volunteer for PLDI'19}
%\cvitem{ASPLOS'21}{External Review Committee Member}
%\cvitem{EuroSys'21}{Shadow Program Committee Member}
%\cvitem{PLDI'20}{Artifact Evaluation Committee Member}
%\cvitem{MLSys'20}{Artifact Evaluation Committee Member}
%\cvitem{ASPLOS'20}{Artifact Evaluation Committee Member}
%\cvitem{SOSP'19}{Artifact Evaluation Committee Member}
%\cvitem{PLDI'19}{Student Volunteer}

%\cvitem{-}{Data Center Consultation for NCC Bank, NRB Commercial Bank, Bangladesh Police.}
%\cvitem{-}{Web Application Security Assessment \& Load Testing for Bangladesh Hajj Pre-registration Server.}
%\cvitem{-}{Organizing Committee Member for NSysS 2016, WALCOM 2015, NSysS 2015.}
%\cvitem{-}{Online Result Processing for Country-wide Election for The Institution of Engineers, Bangladesh (IEB).}
%\cvitem{-}{Fraudulent Transaction Investigation for Sonali Bank Bangladesh.}
%\cvitem{-}{Member of Board of Undergraduate Studies, Self-Assessment Committee, Departmental Security Cell, Under-graduate Admission Programming Committee, Bureau of Research, Testing and Consultation Team, Dept. of Computer Science and Engineering, Bangladesh University of Engineering and Technology.}

\section{Grants}
\cvitem{2020}{Taming the Instruction Bottleneck in Modern Datacenter Applications}
\cvlistitem{Principal Investigators: Baris Kasikci and Joseph Devietti}
\cvlistitem{Organization: NSF/Intel Partnership on Foundational Microarchitecture Research (FoMR)}

\cvitem{2020}{Proxy-Web: A Proxy App Suite for Production Web Services}
\cvlistitem{Principal Investigators: Baris Kasikci, Timothy Rogers, and David Brooks}
\cvlistitem{Organization: Semiconductor Research Corporation (SRC)}

\section{Selected Research Mentoring}
\cvitem{2020}{Dexin Zhang, undergrad, co-author of the ISCA'21 paper}
\cvitem{2019-Present}{Nathan Brown, undergrad, first place in ACM undergraduate student research competition, CGO'20}
\cvitem{2020}{Yuhan Chen, PhD student, results published in IISWC'20}
\cvitem{2018-19}{Yifan Zhao, undergrad (now a PhD student, UIUC), co-author of the PLDI'19 paper}
%\cvitem{Undergraduate}{Yifan Zhao, Nathan Brown, Ashfaqur Rahaman, Yuhan Chen, Yineng Yan, Dexin Zhang, Muhammed Ugur}

\section{Technical Skills}
\cvitem{Languages}{C, C++, Java, x86 Assembly, PL/SQL, Tcl, Python, HTML, CSS, JavaScript, PHP, Shell script, Prolog.}
\cvitem{DB Systems}{PostgreSQL, MySQL, SQLite.}
%\cvitem{Operating Systems}{Variants of Linux and Windows.}
\cvitem{Miscellaneous Tools}{NS-2, NS-3, LLVM, Intel Pin, SimPy, ZSim, Awk, Flex/Bison, OpenGL, Linux perf, Intel PMU tools, Intel PT, nvprof, pgprof, valgrind, Spring, Android, Bootstrap, Arduino, TensorFlow, \LaTeX.}

\end{document}
